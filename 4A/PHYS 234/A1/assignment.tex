\documentclass[12pt]{article}

\usepackage[margin=1in]{geometry}  % set the margins to 1in on all sides
\usepackage{amsmath}               % great math stuff
\usepackage{amsfonts}              % for blackboard bold, etc
\usepackage{braket, units, enumitem}

\begin{document}


\newcounter{set}
\setcounter{set}{1}
\newcounter{problem}[set]
\newcommand{\problem}{{\vspace{2\baselineskip}\noindent\large \bfseries Problem~\arabic{set}:}\\\refstepcounter{set}}
\newcommand{\problemsub}{\refstepcounter{problem}{\vspace{2\baselineskip}\noindent\large \bfseries Problem~\arabic{set}\roman{problem}:}\\}
\newcommand{\problemasub}{\refstepcounter{problem}{\vspace{2\baselineskip}\noindent\large \bfseries Problem~\arabic{set}\alph{problem}:}\\}



\nocite{*}

\title{PHYS 234 - A1}

\author{Alexander Maguire \\ 
amaguire@uwaterloo.ca \\
20396195}

\maketitle

\problem

The photoelectric effect is inexplicable by classical mechanics as to treat light entirely as a wave, its energy must be related to the amplitude of the wave; however, below a certain frequency of light, no degree of intensity could induce a charge in the apparatus. This is explained in quantum mechanics by treating light as photons, whose energy is equal to $h\nu$ (ie. related to the frequency of the light rather than its intensity).

Furthermore, if light existed solely as a wave, at low intensities we should expect to see a time lag between hitting the plate with the photons and it releasing electrons (while the energy accumulates in order to overcome the work function). However, no time lag has ever been observed, implying that all of the light's energy is impacted immediately, leading further credence to the concept of a localized packet of light.

\problem
\begin{align*}
E_{{photon}_{uv}} &= h\nu \\
&= \frac{hc}{\lambda_{uv}} \\
&= \unit[4.966 \times 10^{-19}]{J} \\
\text{likewise, }
E_{{photon}_{ir}} &= \unit[2.838 \times 10^{-19}]{J}
\\ \\
n_{uv} &= \frac{P_{uv}}{E_{{photon}_{uv}}} \\
&= \unit[8.055 \times 10^{19}]{photons/s} \\
\text{likewise, }
n_{ir} &= \unit[1.409 \times 10^{20}]{photons/s}
\end{align*}

\newpage
\begin{enumerate}[label=\alph{*})]
\item Because $n_{ir} > n_{uv}$, the infrared bulb emits more photons per unit time.
\item $n_{ir} - n_{uv} = \unit[6.035 \times 10^{19}]{photons/s}$
\end{enumerate}

\problem

\begin{align*}
V &= m\left(\frac{c}{\lambda} - \nu_0\right) \\
\to m &= \frac{V}{\frac{c}{\lambda} - \nu_0} \\
\\
\to \nu_0 &= \frac{\frac{V_1 c}{\lambda_2} - \frac{V_2 c}{\lambda_1}}{V_1 - V_2} \\
&= \frac{\frac{c \cdot \unit[1.85]{V}}{\unit[4000]{\AA}} - \frac{c \cdot \unit[0.82]{V}}{\unit[3000]{\AA}}}
{\unit[1.85]{V} - \unit[0.82]{V}} \\
&= \unit[5.5 \times 10^{14}]{Hz} \\
\\
\to m &= \frac{\unit[1.85]{V}}{\frac{c}{\unit[3000]{\AA}} - \nu_0} \\
&= \unit[4.12 \times 10^{-15}]{V/Hz} \\
\\
\to V_0 &= -m\nu_0\\
&= \unit[-2.26]{V}
\end{align*}

\begin{enumerate}[label=\alph{*})]
\item 
\begin{align*}
eV_0 &= -h\nu_0 \\
\to h &= \frac{eV_0}{-\nu_0} \\
&= \unit[-7 \times 10^{-34}]{J\cdot s}
\end{align*}

\item $w_0 = h\nu_0 = \unit[2.28]{eV}$

\item $\lambda_{Na} = c/\nu_0 = \unit[5440]{\AA}$
\end{enumerate}

\newpage

\problem
Given $\Delta \lambda = \lambda_c (1 - cos \theta)$, we see $\max \Delta \lambda = 2\lambda_c$ occurs at $\theta = n\pi$ for $n \in \{1,3,5...\}$.

\begin{align*}
%-\frac{dE}{d\lambda} &= h\frac{c}{\lambda^2} = \frac{1}{2}m_ev_{max}^2 \\
E = h\frac{c}{\lambda} &= K_{max} = \frac{1}{2}m_ev_{max}^2 \\
\to \frac{v_{max}}{c} &= \sqrt{\frac{2h}{cm_e\lambda}}\biggr\rvert_{\lambda = \unit[482.4]{pm}} \\
&= 0.1003
\end{align*}

\problemsub
$$z^* = 3 + 4i$$

\problemsub
\begin{align*}
||z|| &= \sqrt{\braket{z|z}} \\
&= \sqrt{3^2+4^2} \\
&= 5
\end{align*}

\problemsub
\begin{align*}
\theta &= \arctan{\frac{\Im{z}}{\Re{z}}} \\
&= \arctan{\frac{-4}{3}} \\
&= \unit[-0.92729]{rad}
\\ \\
z &= re^{i\theta} \\
&= ||z|| \cdot e^{i\theta} \\
&= 5e^{-0.92729i}
\end{align*}

\newpage

\problemsub
\begin{align*}
z &= r(\cos{\theta} + i\sin{\theta}) \\
&= 5[\cos(-0.92729) + i\sin(-0.92729)]
\end{align*}

\stepcounter{set}
\problem

\centering{
\begin{tabular}{ r | c c c }
  $\Braket{x|y}$ & $\Ket{\alpha_1}$ & $\Ket{\alpha_2}$ & $\Ket{\alpha_3}$ \\ \hline
  $\Bra{\alpha_1}$ & $2$ & $1 - i$ & $2 - i$ \\
  $\Bra{\alpha_2}$ & $1 + i$ & $2$ & $2 + i$ \\
  $\Bra{\alpha_3}$ & $2+i$ & $2 - i$ & $3$ \\
\end{tabular}
}





\end{document}