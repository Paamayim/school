\documentclass[12pt]{article}

\usepackage[margin=1in]{geometry}
\usepackage{amsmath}
\usepackage{amsfonts}
\usepackage{braket, units, enumitem}
\usepackage{todonotes}
\usepackage{graphicx}

\begin{document}

\newcounter{set}
\setcounter{set}{1}
\newcounter{sub}
\setcounter{sub}{1}
\newcounter{problem}[set]
\newcounter{problemi}[sub]
\newcommand{\problem}{{\vspace{2\baselineskip}\noindent\large \bfseries Problem~\arabic{set}:}\\\refstepcounter{set}}
\newcommand{\problemsub}{\setcounter{problemi}{0}\refstepcounter{problem}{\vspace{2\baselineskip}\noindent\large \bfseries Problem~\arabic{set} \roman{problem}:}\\}
\newcommand{\problemasub}{\setcounter{problemi}{0}\refstepcounter{problem}{\vspace{2\baselineskip}\noindent\large \bfseries Problem~\arabic{set}\alph{problem}:}\\}
\newcommand{\problemasubi}{\refstepcounter{problemi}{\vspace{2\baselineskip}\noindent\large \bfseries
Problem~\arabic{set}\alph{problem} \roman{problemi}:\\}}
\newcommand{\fig}[3]{
    \begin{center}
    \includegraphics[scale=#3]{#1} \\
    #2 \\
    \end{center}
%    \label{fig:awesome_image}
}


\nocite{*}

\title{CS 486 - A2}

\author{Alexander Maguire \\
amaguire@uwaterloo.ca \\
20396195}

\maketitle

\problemasub
Paths are expanded in the order:

S F P Q R T G

\problemasubi
Yes, $h(\textit{state})$ is admissible since it never overestimates the cost of getting to $g$.

\problemasubi
S H K C A B D M G

\problemasubi
S H K C F P Q R T G




\stepcounter{set}
\problemasub
Consider a node to be a partial tour, with the first city always being $A$. Neighbors can be generated by appending an
unvisited city to the end of the partial tour, with a goal state being one node with $n$ cities visited, where $n$ is
the number of cities. A cost function between any two nodes is simply the distance of moving from the current city to
the unvisited city, with an extra cost for completing the tour (to get back to the starting city).

\problemasub
After failing to come up with my own heuristic, some googling lead to the idea of using a minimal spanning tree of the
unvisited cities as a heuristic. The page also described adding the costs from the current city to the nearest unvisited
city, as well as the cost from the starting node to the nearest unvisited city. However, when I attempted to implement
this, my heuristic would sometimes overestimate the total tour cost, resulting in incorrect solutions.

My final heuristic was simply to use the MST of the unvisited cities; this worked well.

\problemasub
TODO: graph this
TODO: talk about 36 problem



\stepcounter{set}
\problemasub
To generate the neighbors of a node, I randomly swapped three cities in a tour (ensuring that the first city was never
swapped. This is guaranteed to preserve a tour, since swapping any three cities in a tour will result in another tour.

\problemasub
My annealing schedule was made up of five parameters:
\begin{itemize}
    \item $\alpha = $ \textbf{initTemp}: the starting temperature
    \item $\Omega = $ \textbf{minTemp}: the stopping temperature
    \item $\beta = $ \textbf{coolingFactor}: the rate at which the temperature is cooled
    \item $\Gamma = $ \textbf{itersPerTemp}: how many iterations the temperature remains constant
    \item $g = $ \textbf{isGeometric}: whether the cooling is geometric or arithmetic
\end{itemize}

Having taken a class on local-search last semester, I reused three of my best cooling schedules:
\begin{enumerate}
    \item $\alpha = 13\quad \Omega = 0.01\quad \beta = 0.01\quad \Gamma = 3\quad g = 0$
    \item $\alpha = 15\quad \Omega = 0.0002\quad \beta = 0.93\quad \Gamma = 150\quad g = 1$
    \item $\alpha = 12\quad \Omega = 0.00001\quad \beta = 0.93\quad \Gamma = 250\quad g = 1$
\end{enumerate}

And I found that (1) worked the best for small ($n < 5$) datasets, (2) for medium ($5 \leq n < 15$) datasets, and (3)
for large($15 \leq n$) datasets. However, since (3) performed quite well on all datasets, I decided to use that as my
chosen annealing schedule.

\problemasub
\fig{sa.png}{SA tour cost over time}{0.5}
The best solution I found was ``AJQZXVPOADECACSUFGBAJHAGAERMAAAFTYIAIAHLDNABWK'', with tour cost 511.12.

\problemasub
Simulated annealing is not complete, since it is not guaranteed to find the best answer (merely a \textit{good} answer).

\problemasub
Simulated annealing is also not optimal, since it's ability to search worse neighborhoods -- while good for escaping
local maxima -- will usually jump to a neighborhood with significantly worse options.



\end{document}
