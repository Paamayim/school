\documentclass[12pt]{article}

\usepackage[margin=1in]{geometry}
\usepackage{amsmath}
\usepackage{amsfonts}
\usepackage{braket, units, enumitem}
\usepackage{todonotes}

\begin{document}

\newcounter{set}
\setcounter{set}{1}
\newcounter{sub}
\setcounter{sub}{1}
\newcounter{problem}[set]
\newcounter{problemi}[sub]
\newcommand{\problem}{{\vspace{2\baselineskip}\noindent\large \bfseries Problem~\arabic{set}:}\\\refstepcounter{set}}
\newcommand{\problemsub}{\setcounter{problemi}{0}\refstepcounter{problem}{\vspace{2\baselineskip}\noindent\large \bfseries Problem~\arabic{set} \roman{problem}:}\\}
\newcommand{\problemasub}{\setcounter{problemi}{0}\refstepcounter{problem}{\vspace{2\baselineskip}\noindent\large \bfseries Problem~\arabic{set}\alph{problem}:}\\}
\newcommand{\problemasubi}{\refstepcounter{problemi}{\vspace{2\baselineskip}\noindent\large \bfseries
Problem~\arabic{set}\alph{problem} \roman{problemi}:\\}}


\nocite{*}

\title{CS 486 - A2}

\author{Alexander Maguire \\
amaguire@uwaterloo.ca \\
20396195}

\maketitle

\problemasub
Paths are expanded in the order:

S F P Q R T G

\problemasubi
Yes, $h(\textit{state})$ is admissible since it never overestimates the cost of getting to $g$.

\problemasubi
S H K C A B D M G

\problemasubi
S H K C F P Q R T G

               4+4a 5+3b
               3+3c 6+2d
1+5f 0+4s 1+3h 2+2k 7+1m 2e
2+4p 3+3q 4+2r 5+1t 6+0g 1n


\end{document}
