\documentclass[12pt]{article}

\usepackage[margin=1in]{geometry}
\usepackage{amsmath, amssymb}
\usepackage{amsfonts}
\usepackage{braket, units, enumitem}
\usepackage{todonotes}
\usepackage{graphicx}
\usepackage{mdframed}

\newmdenv[
  topline=false,
  bottomline=false,
  rightline=false,
  skipabove=\topsep,
  skipbelow=\topsep
]{leftrule}

\usepackage{tikz}
\newcommand*\circled[1]{\tikz[baseline=(char.base)]{
            \node[shape=circle,draw,inner sep=2pt] (char) {#1};}}
\let\oldBC\because
\renewcommand\because{\raisebox{0.75pt}{$\quad\oldBC\quad$}}

\newcommand{\fig}[3]{
    \begin{center}
    \includegraphics[scale=#3]{#1} \\
    #2 \\
    \end{center}
}

\begin{document}

\newcounter{set}
\setcounter{set}{1}
\newcounter{problem}[set]
\newcommand{\problem}{{\vspace{2\baselineskip}\noindent\large \bfseries Problem~\arabic{set}:}\\\refstepcounter{set}}
\newcommand{\problemsub}{\refstepcounter{problem}{\vspace{2\baselineskip}\noindent\large \bfseries Problem~\arabic{set} \roman{problem}:}\\}
\newcommand{\problemasub}{\refstepcounter{problem}{\vspace{2\baselineskip}\noindent\large \bfseries Problem~\arabic{set}\alph{problem}:}\\}

\title{CS 486 - A3}

\author{Alexander Maguire \\
amaguire@uwaterloo.ca \\
20396195}

\maketitle

\problemsub

\problemsub

\stepcounter{set}
\problem
\newcommand\learn[2]{\textbf{From} \ensuremath{#1} $\to$ \textbf{Learn} \ensuremath{#2}}
\newcommand\try[2]{\textbf{Try} \ensuremath{#1} \begin{leftrule}#2\end{leftrule}}
\newcommand\cdict[1]{\textbf{Contradiction} \ensuremath{#1} $\to$ \textbf{Backtrack}}

Given:
\begin{alignat*}{2}
    &\circled{1}: &S+S &= Y + 10 X_1 \\
    &\circled{2}: &E+E+X_1 &= R + 10 X_2 \\
    &\circled{3}: \quad&O+N+X_2 &= A + 10 X_3 \\
    &\circled{4}: &R+O+X_3 &= N + 10 X_4 \\
    &\circled{5}: &E+X_4 &= I + 10 X_5 \\
    &\circled{6}: &Z+X_5 &= B \\
\end{alignat*}

\noindent

\begin{enumerate}
\item \learn{\circled{1}}{S \neq Y \neq 0}
\item \learn{\circled{6}}{Z \neq B}
\item \learn{Z \neq B \land \circled{6}}{X_5 = 1}
\item \learn{\circled{5} \land X_5 = 1}{X_4 = 1} \because $E \leq 9$
\item \learn{X_4 = 1 \land \circled{5}}{E = 9, I = 0}
\item \learn{\circled{2}\big|_{E = 8} \land R \neq 9}{X_1 = 0, X_2 = 1, R = 8}
\item \learn{\circled{1} \land X_1 = 0 \land Y \leq 8}{S \in \{1,2,3\}} \because $Y \neq 8$
\item \learn{\circled{3}}{N \in \{1,2,3,4\}} \because $O \in \{N+2, N+3\} \land O \not\in \{8,9\}$
\item \try{S=1} {
    \begin{enumerate}
    \item \learn{\text{9.}}{Y = 2}
    \item \learn{\text{9.}}{N \in \{3, 4\}}
    \item \try{N=3} {
        \begin{enumerate}
        \item \learn{\text{(c)}}{O \in \{5,6\}}
        \item \try{O=5} { \cdict{\circled{3}} }
        \item \try{O=6} { \cdict{\circled{3}} }
        \item \cdict{O \not\in \{5,6\}}
        \end{enumerate}
    }
    \item \try{N=4} {
        \begin{enumerate}
        \item \learn{\text{(d)}}{O \in \{6,7\}}
        \item \try{O=6} { \cdict{\circled{3}} }
        \item \try{O=7} { \cdict{\circled{3}} }
        \item \cdict{O \not\in \{6,7\}}
        \end{enumerate}
    }
    \end{enumerate}
    \item \cdict{S \neq 1}
}
\item \try{S=2} {
    \begin{enumerate}
        \item \learn{\text{10.}}{Y = 4}
        \item \learn{\text{10.}}{N \in \{1,3\}}
        \item \try{N=1} {
            \begin{enumerate}
            \item \learn{\text{(c)}}{O = 3}
            \item \learn{\circled{3}}{A = 5}
            \item \learn{\circled{6}}{Z = 6, B = 7}
            \item \textbf{Success!}
            \end{enumerate}
        }
    \end{enumerate}
}
\end{enumerate}

Final solution: $I = 0, N = 1, S = 2, O = 3, Y = 4, A = 5, Z = 6, B = 7, R = 8, E = 9$.


\problemsub

\problemsub

\stepcounter{set}
\problemasub

Let $\textit{cell}_{ij}$ be the value of the cell in column $i$, row $j$, with domain $[0, 9)$.

The constraints for sudoku are thus the union of $\textit{Alldiff}$ of the following sets over all combinations of
$i, j \in [0, 9)$:

\begin{align*}
    \texttt{col}_i &= \bigcup_j \texttt{cell}_{ij} \\
    \texttt{row}_j &= \bigcup_i \texttt{cell}_{ij} \\
    \texttt{square}_{xy} &= \bigcup_{\substack{3x \leq i < 3x+3 \\ 3y \leq j < 3y+3}}{\texttt{cell}_{ij}}
\end{align*}

\problemasub

The code for this problem is included in \texttt{Sudoku.scala}.

\problemasub

\textbf{Figure 1} plots the average cell assignments while solving a board of sudoku against the number of given cells.
The results are pretty good -- in most cases the number of assignments is very close to the number of open cells.
However, this number jumps up by an order of magnitude in the mid-20s, likely due to this being an equilibrium between
being under- and over-constrained.

Additionally, though it is not visible on the plot, problem \texttt{18/1.sd} took 15481 assignments. The solution was
found quickly, but the assignment explicitly says not to include results over 10000 assignments. My assumption about the
poor runtime of this test-case is that it happens to be quite antagonistic to my heuristic.

\fig{sudokuplot.png}{\textbf{Figure 1:} Average cell assignments per number of given cells}{0.5}







\end{document}
