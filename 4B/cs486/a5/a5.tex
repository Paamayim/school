\documentclass[12pt]{article}

\usepackage[margin=1in]{geometry}
\usepackage{amsmath, amssymb}
\usepackage{amsfonts}
\usepackage{braket, units, enumitem}
\usepackage{todonotes}
\usepackage{graphicx}
\usepackage{mdframed}
\usepackage{tikz}
\usepackage{tikz-cd}
\usetikzlibrary{decorations.pathmorphing}
\usetikzlibrary{matrix}


\newmdenv[
  topline=false,
  bottomline=false,
  rightline=false,
  skipabove=\topsep,
  skipbelow=\topsep
]{leftrule}

\usepackage{tikz}
\newcommand*\circled[1]{\tikz[baseline=(char.base)]{
            \node[shape=circle,draw,inner sep=2pt] (char) {#1};}}
\let\oldBC\because
\renewcommand\because{\raisebox{0.75pt}{$\quad\oldBC\quad$}}

\newcommand{\fig}[3]{
    \begin{center}
    \includegraphics[scale=#3]{#1} \\
    #2 \\
    \end{center}
}

\begin{document}

\newcounter{set}
\setcounter{set}{1}
\newcounter{problem}[set]
\newcommand{\problem}{{\vspace{2\baselineskip}\noindent\large \bfseries Problem~\arabic{set}:}\\\refstepcounter{set}}
\newcommand{\problemsub}{\refstepcounter{problem}{\vspace{2\baselineskip}\noindent\large \bfseries Problem~\arabic{set} \roman{problem}:}\\}
\newcommand{\problemasub}{\refstepcounter{problem}{\vspace{2\baselineskip}\noindent\large \bfseries Problem~\arabic{set}\alph{problem}:}\\}

\title{CS 486 - A5}

\author{Alexander Maguire \\
amaguire@uwaterloo.ca \\
20396195}

\setlength{\parindent}{0pt}
\twocolumn
\maketitle

\problemasub

The code to generate this section is included as \texttt{policy.hs}. \\

\begin{tabular}{cccc}
    206 $\leftarrow$ & 147 $\leftarrow$ & 105 $\leftarrow$ \\
    147 $\uparrow$ & 105 $\uparrow$ & 74 $\uparrow$ \\
    116 $\uparrow$ & 82 $\uparrow$ & 58 $\uparrow$
\end{tabular}

\problemasub
\begin{tabular}{ccc}
    6 $\rightarrow$ & 4 $\rightarrow$ & 2 $\downarrow$ \\
    12 $\rightarrow$ & 8 $\uparrow$ & 4 $\uparrow$ \\
    19 $\uparrow$ & 12 $\uparrow$ & 8 $\uparrow$
\end{tabular}

\problemasub
\begin{tabular}{ccc}
    9 $\rightarrow$ & 5 $\rightarrow$ & 3 $\downarrow$ \\
    12 $\uparrow$ & 8 $\uparrow$ & 4 $\uparrow$ \\
    19 $\uparrow$ & 12 $\uparrow$ & 8 $\uparrow$
\end{tabular}

\problemasub
\begin{tabular}{ccc}
    12 $\rightarrow$ & 7 $\rightarrow$ & 4 $\downarrow$ \\
    12 $\uparrow$ & 8 $\uparrow$ & 4 $\uparrow$ \\
    19 $\uparrow$ & 12 $\uparrow$ & 8 $\uparrow$
\end{tabular}



\newpage
\stepcounter{set}
\problem

\begin{tabular}{cccc}
    {}   & N      & C      & J      \\
    N    & 73, 25 & 57, 42 & 66, 32 \\
    C    & 80, 26 & 35, 12 & 32, 54 \\
    J    & 28, 27 & 63, 31 & 54, 29 \\
\end{tabular} \vspace{5mm} \\

\begin{tabular}{cccc}
    {}   & C      & J      \\
    N    & 57, 42 & 66, 32 \\
    C    & 35, 12 & 32, 54 \\
    J    & 63, 31 & 54, 29 \\
\end{tabular} \vspace{5mm}

\begin{tabular}{cccc}
    {}   & C      & J      \\
    N    & 57, 42 & 66, 32 \\
    J    & 63, 31 & 54, 29 \\
\end{tabular} \vspace{5mm}

\begin{tabular}{cccc}
    {}   & C      \\
    N    & 57, 42 \\
    J    & 63, 31 \\
\end{tabular} \vspace{5mm}

\begin{tabular}{cccc}
    {}   & C      \\
    J    & 63, 31 \\
\end{tabular} \vspace{5mm} \\

This is a Nash equilibrium, because given $(J,C)$, player 1 can't do better by changing from $J$, and player 2 can't do
better by changing from $C$.


\newpage
\problem

Leave-one-out cross-validation is the process of partitioning your dataset of $n$ elements into a training set with
$n-1$ elements and a testing set of with 1 element. The number of positive examples is the same as the number
of negative examples in the original dataset.

Without loss of generality, assume the sample taken for the testing dataset is classified positive, so its majority is
positive. The training set now contains $n/2 - 1$ positive examples, and $n/2$ negative examples, so its majority is
negative. Leave-one-out cross-validation will now always score zero by the construction of the training set.

\problemsub

\problemsub

\problemsub

\problemsub

\problemsub

\problemsub

\stepcounter{set}
\problemsub

\problemsub

\problemsub

\problemsub

\problemsub

\end{document}
