\documentclass{article}

\usepackage{color,soul,todonotes,hyperref}

\title{On Automation Makes Us Dumb}
\author{Alexander Maguire, \#20396195}

\begin{document}

\maketitle

Carr's article \emph{Automation Makes Us Dumb} presents a controversial argument that modern society's tendency to
automate as much of life as possible is actively harmful both to society and the individual humans of whom it is
comprised.  He argues that as we delegate more and more of our day-to-day jobs to computers, the technologies stop being
tools to help, and instead coerce those using them into \emph{operators}. ``More often than not,'' Carr writes, ``the
new machines were leaving workers with drabber, less demanding jobs. An automated milling machine, for example, didn’t
transform the metalworker into a more creative artisan; it turned him into a pusher of buttons.\cite{carr}'' As a
result, humans are becoming less skilled and increasingly incapable of handling situations when the computer systems
fail.

In a way, Carr's depiction of technology-centered automation can be viewed as an interesting subversion of McCulloch's
original vision of mechanization. One of McCulloch's major contributions to the world was in the form of the artificial
(MCP) neuron -- an early building block in the attempt to create human-like machines\cite{mcculloch}. Contemporary approaches to
automation are quite ironic in this respect, as they seem to instead be creating \emph{machine-line humans}. Carr
again: ``As software improves, the people using it become less likely to sharpen their own know-how. Applications that
offer lots of prompts and tips are often to blame; simpler, less solicitous programs push people harder to think, act
and learn.\cite{carr}''

In other words, Carr seems to be saying, our ``smarter'' technology is reducing our ability to behave as the rational
animals we purport to be; instead we find ourselves delegating away our abilities to problem solve.  Carr alludes that
modern software, with all of its prompts and dialogs, is steadily transforming humans into nothing but a step in an
algorithm -- one perhaps labeled as ``insert creativity here''. Originally automation was intended to replace dreary
work, but we now find ourselves using it to replace \emph{all work} seemingly without questioning whether or not this is
what we want or need.

However dreadful this trend of mechanization might seem, Carr reminds us that not all is lost. In a concept that he
refers to as ``human-centered automation'' -- a state of automation where the computer acts more like a personal trainer
than a foreman. One exciting example Carr provides, is software which is capable of paying attention to ones performance
and suggesting (or silently performing) context shifts when the user becomes frustrated or stuck. Carr suggests a path
that combines the best of both worlds: optimizing work efficiency without relegating humans to modern salt mines.

Human-centered automation can be seen as a return to some early ideas of cybernetics in that it refocuses on the Wiener
and Ashby's original notion of feedback being critical for cybernetics\cite{ashby56}\cite{wiener}. Wiener and Ashby's
work shows a distinctly strong emphasis on feedback as being crucial to the process. However, historically, this
feedback has been viewed as belonging to a separate magisterium from humanity. Human-centered automation refocuses on
this emphasis by introducing humans into the feedback loop; instead of being merely a creative (or other domains in
which humans still have a comparative advantage over machines) node in a flow-chart, human talent and design-work is
augmented by the automation rather than \emph{being replaced by it}.

Personally, I think Carr's example of adaptive automation achieves the stated goals of human-centered automation. As a
software engineer, I believe my field to be on the cutting edge of this approach to automation (and likely other
approaches as well). In fact, my current capstone design project is one which silently monitors a team's working habits
while developing software. It stays out of the way for the most part, but when prompted can provide two major benefits:
it maintains a top-down dashboard view stating \emph{who knows what} on the project (a surprisingly hard problem in
collaborative environments). Additionally, our software can help users find what they're looking for when they become
stuck in areas they don't know about, by analyzing what others were doing while working in the same area. In some of our
initial tests with users, after people have become familiar with how to use it, their productivity has improved by an
average of 20\%, with a significant decrease in self-reported frustration.

I honestly think this emerging new push towards human-centered automation is one of the best paradigm shifts of
computing over the last two decades. As a user, I would be ecstatic to have a cohesive human-centered approach for all
of computing, especially as computing becomes more ubiquitous and less centered around screens. As the discovery channel
once put it, ``the future's pretty cool.''


\begin{thebibliography}{1}
    \bibitem{carr}
        Carr, N. (2014, Nov). Automation Makes Us Dumb. Wall Street Journal. Retrieved from \\
        \url{http://www.wsj.com/articles/automation-makes-us-dumb-1416589342}
    \bibitem{mcculloch}
        McCulloch W. (1965) Embodiments of Mind. MIT Press.
    \bibitem{ashby56}
        Ashby, R. (1956). Introduction to Cybernetics. Chapman \& Hall.
    \bibitem{wiener}
        Wiener, N. (1948). Cybernetics or Control and Communication in the Animal and the Machine. Hermann et Cie. MIT Press.
\end{thebibliography}

\end{document}

